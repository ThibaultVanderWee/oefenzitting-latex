\documentclass[12pt, dutch]{article}

\usepackage[multiple]{footmisc}
\usepackage{babel}
\usepackage[final]{graphicx}
\usepackage[hidelinks, hyperfootnotes=false]{hyperref}
\usepackage{fullpage}
\usepackage{textcomp}
\usepackage[utf8]{inputenc}
\usepackage[T1]{fontenc}
\usepackage{lmodern}
\usepackage{dtklogos}
\usepackage{microtype}

\usepackage{tcolorbox}

\usepackage{cleveref}
\crefname{figure}{Figuur}{Figuur}

\usepackage{tikz}
\usetikzlibrary{positioning, arrows, calc}

\newcommand{\bs}{\textbackslash}

\hyphenation{ge-lijk-aardige}

\title{\LaTeX: Eerste oefenzitting}
\author{Informatica werktuigen}
\date{}

\begin{document}

\maketitle

\section{\LaTeX{}}

\LaTeX{} (spreek uit: La-Tech, omdat de laatste X eigenlijk een
Griekse chi ($\chi$) is) wordt uitgevoerd vanaf de prompt, dus in een
terminal.\footnote{Onder Windows: het startmenu openenen en \texttt{cmd.exe} invoeren.}
Er bestaan ook gebruiksvriendelijkere omgevingen, maar die hebben meestal het
nadeel dat de gebruiker niet goed kan inschatten welke fout hij gemaakt heeft.

In tegenstelling tot een WYSIWYG\footnote{What You See Is What You
  Get, of What You See Is All You've Got} tekstverwerker werkt
\LaTeX{} met een bronbestand, in platte tekst, dat gecompileerd wordt.
Deze platte tekst is uiteraard voorzien van tags, net zoals een HTML
bestand.

Het resultaat van het bronbestand na verwerking door \LaTeX{} kun je op
verschillende manieren bekijken. Je kunt kiezen om \LaTeX{} te compileren naar
een PDF, of een aantal andere uitvoerbestanden, zoals DVI of Postscript.
Aangezien die andere formaten niet meer zo vaak gebruikt worden, kijken we enkel
naar PDF. Stel dat je \LaTeX-bronbestand \texttt{tekst.tex} heet, gebruiken we
\texttt{pdflatex tekst.tex} om dit bestand te compileren naar een PDF. Je kunt
dit bestand bijvoorbeeld openen met \texttt{evince}, maar je kunt ook het
algemene commando \texttt{xdg-open tekst.pdf} gebruiken, waarbij Linux een
applicatie kiest die dit bestand kan openen.

\LaTeX{}-bronbestanden bewerk je best in een editor die \LaTeX{}-syntax-
highlighting ondersteunt, om zo het werk te vergemakkelijken. Bijna elke
Linux editor heeft standaard syntax highlighting, al dan niet mits de
optie expliciet aan te zetten, bvb.\ \textbf{gedit} of \textbf{kile}.
Ook voor Windows zijn er veel
gratis\footnote{\url{http://www.latexeditor.org/}}\footnote{\url{http://www.texniccenter.org/}}
en betalende\footnote{\url{http://www.winedt.com/}} applicaties om
\LaTeX{}-bestanden te bewerken.

\subsection{Introductie tot \LaTeX}
Open een teksteditor naar keuze en open het bestand \texttt{opgave1.tex}.
Om dit bestand te compileren open je een console en voer je de nodige opdrachten
in.
\begin{tcolorbox}[title=Opgave 1]
  Compileer het bestand opgave1.tex en open de pdf.
\end{tcolorbox}

Indien het bestand een fout bevat, zal \LaTeX{} overschakelen naar een
interactieve modus waarmee je de fout kan verbeteren. Typ \texttt{X} om deze te
verlaten.

\begin{tcolorbox}[title=Opgave 2]
  Compileer het bestand opgave2.tex. Zoals je merkt staat er een fout in dit
  bestand. Los de fout op en compileer het bestand opnieuw. Vergelijk indien nodig
  met het bestand van opgave 1 om uit te vinden waar de fout ligt.
\end{tcolorbox}
Het is eenvoudiger om \LaTeX{} te laten werken in nonstop-mode. Dit doe je als volgt:
\begin{center} \verb'pdflatex -interaction=nonstopmode tekst.tex' \end{center}
Op deze manier zal \LaTeX{} niet voor iedere fout de compilatie onderbreken.



Gemakkelijker is om een editor te gebruiken die rechtstreeks \LaTeX{} ondersteunt,
en waarmee je kunt compileren en visualiseren met \'e\'en toetsaanslag.
Vermits dit afhankelijk is van het OS (Windows/Linux/Mac OS X) en de gebruikte
editor, zoek je best zelf uit welke editor voor jou het beste werkt. Tijdens
deze oefenzitting leren we jullie nog iets meer over hoe \LaTeX{} compileert, en
zullen we dus soms toch vragen om het \texttt{pdflatex} commando te gebruiken.

\subsection{De preambule}
We gaan nu meer in detail bekijken wat er juist in de broncode van het bestand
staat. Bekijk de opties die gegeven zijn in de \texttt{\bs{}documentclass}
opdracht op de eerste regel van \texttt{opgave3.tex}. De optie \texttt{a4paper} verzekert je ervan dat
\LaTeX{} de tekst zal formatteren op een A4 blad. Anders bestaat de kans dat
het Amerikaanse \texttt{letter} formaat gebruikt zal worden.

In de preambule\footnote{Dit is het deel vóór het eigenlijke
  document, zeg maar de instellingen van het document.} vind je ook
een \texttt{\bs{}usepackage} commando terug. Dit commando laadt het
\texttt{babel} pakket. Dit \texttt{babel} pakket verzorgt taalafhankelijkheden
tijdens het verwerken van de tekst. \LaTeX{} gaat namelijk zelf
woorden splitsen en de splitsingsregels verschillen van taal tot
taal. Sectietitels en hoofdstuktitels veranderen ook als je een
inhoudstabel genereert zal \LaTeX{} die zelf de titel ``Table of
Contents'' of ``Inhoudstabel'' geven naargelang de taal die je
gebruikt. Het argument dat je mee moet geven is de taal die gebruikt
wordt in de tekst. In het voorbeeld is dit Nederlands (\texttt{dutch}).


\begin{tcolorbox}[title=Opgave 3]
  Laad het \texttt{todonotes} pakket in \texttt{opgave3.tex}. Zonder dit pakket zal de opgave niet
  succesvol compileren. Compileer het bestand en open de resulterende PDF.
\end{tcolorbox}


\begin{tcolorbox}[title=Opgave 4a]
  Compileer \texttt{opgave4.tex} en open de resulterende PDF.
\end{tcolorbox}

Het gegenereerde PDF-document ziet er op het eerste zicht in orde uit.
Toch willen we het document verbeteren door een titelpagina toe te voegen.

% \subsection{Het \texttt{\bs{}documentclass} commando}

\begin{tcolorbox}[title=Opgave 4c]
  Voeg aan de opties voor het \texttt{\bs{}documentclass} commando de
  optie \texttt{titlepage} toe.  De opties voor een \LaTeX{} commando
  moeten van elkaar gescheiden worden aan de hand van komma's. Tip: de opties bevinden zich binnen de vierkante haakjes, oftewel \textit{brackets} in het Engels. Wat
  heeft dit voor effect?  Wat gebeurt er als je de optie \texttt{12pt}
  toevoegt?
\end{tcolorbox}
\begin{tcolorbox}[title=Opgave 4d]
  Wijzig de klasse van het document van \texttt{article} naar
  \texttt{report}, verander alle \texttt{\bs{}section} commando's naar
  \texttt{\bs{}chapter} commando's en alle \texttt{\bs{}subsection} naar
  \texttt{\bs{}section} commando's. Wat verardert dit aan het document?
\end{tcolorbox}

\subsection{Tabellen}

In de derde paragraaf ``De derde fase'', heeft men het over de
kalender voor de invoering van de euro. We gaan deze kalender in een
tabel zetten. Hiervoor zullen we het commando \texttt{\bs{}tabular}
gebruiken. Zoek met Google hoe dit commando werkt.
\begin{tcolorbox}[title=Opgave 5a]
  Wijzig de tekst zodat de kalender in tabelvorm in de tekst terug te vinden is.
\end{tcolorbox}
Zorg ervoor dat de datum gecentreerd staat en zorg ervoor dat de
tekst in de tweede kolom niet buiten de kantlijnen loopt. Je doet dit
door voor de tweede kolom een lengte op te geven: gebruik het
\texttt{p\{lengte\}} commando in de parameterlijst van de
tabel (waar je normaal de letter \texttt{l} zou gebruiken om de kolom
links uit te vullen). Een \texttt{lengte} is een getal gevolgd door een
lengtemaat, bv.\ \texttt{5cm}.

\begin{tcolorbox}[title=Opgave 5b]
  Zorg ervoor dat de tabel gecentreerd is (met \texttt{\bs{}begin\{center\}} en
  \texttt{\bs{}end\{center\}}).
\end{tcolorbox}

Het resultaat zou er ongeveer als volgt moeten uitzien:

\begin{center}
  \begin{tabular}{|c|p{7cm}|}
    \hline
    01.01.1999 & giraal geldverkeer in de nationale munten en in euro                     \\ \hline
    01.01.2002 & invoering van biljetten en muntstukken, giraal geldverkeer enkel in euro \\ \hline
    01.07.2002 & (ten laatste): alles in euro!                                            \\ \hline
  \end{tabular}
\end{center}

\subsection{Lijsten}

Heb je ook gemerkt dat de criteria van Maastricht op het zicht
moeilijk te herkennen zijn in de tekst? In een goede tekst moet
belangrijke informatie duidelijk herkenbaar zijn via de layout. Daar
moeten we dus iets aan veranderen. \LaTeX{} voorziet hiervoor de
\texttt{description} lijstomgeving. Deze omgeving laat je toe om van
elk element in een lijst van beschrijvingen aan te geven welke term je
beschrijft. Je kan dit bij elk \texttt{item} als een optioneel argument
(tussen vierkante haken: \texttt{[]}) meegeven. Een voorbeeld:
\begin{verbatim}
\begin{description}
\item[regen] is nogal nat en valt uit de hemel
\item[sneeuw] is nogal wit en valt uit de hemel
\end{description}
\end{verbatim}

Het resultaat ziet er als volgt uit:
\begin{description}
  \item[regen] is nogal nat en valt uit de hemel
  \item[sneeuw] is nogal wit en valt uit de hemel
\end{description}

\begin{tcolorbox}[title=Opgave 5c]
  Zoek de documentatie op van de \texttt{description}-omgeving en
  gebruik deze voor de 4 criteria van Maastricht: Prijsstabiliteit,
  Wisselkoersen, Rentetarieven en Overheidsschulden.
\end{tcolorbox}

\subsection{Figuren}

De term figuur in \LaTeX{} betekent niet hetzelfde als een afbeelding.
Een figuur moet je eigenlijk zien als een kader met een bepaalde
illustratie in (bijvoorbeeld een afbeelding, maar ook bijvoorbeeld een
tabel of een citaat), die niet tussen de tekst zelf geplaatst wordt,
maar ergens als een apart deel in het document ``rondzweeft'' (in het
Engels spreekt men soms van ``floating''). \LaTeX{} gebruikt bepaalde
algoritmes om een ideale plaatsing te bepalen, maar indien echt nodig
kan je dit op bepaalde manieren be\"{\i}nvloeden.

\begin{tcolorbox}[title=Opgave 5d]
  Voeg een drijvende figuur toe. Hiervoor gebruik je de
  \texttt{figure}-omgeving. Plaats hier de tekst ``voorbeeldfiguur'' in.
\end{tcolorbox}


Je kunt in de \texttt{figure} omgeving ook een ondertekst geven aan de figuur
aan de hand van het \texttt{\bs{}caption} commando. Het \texttt{\bs{}caption}
commando bevat als argument gewoon de ondertekst. Je kunt
de figuur een label geven aan door het \texttt{\bs{}label} commando te
gebruiken na het \texttt{\bs{}caption} commando. Dit
label is niet zichtbaar in de gegenereerde PDF, maar vanuit andere
delen van de tekst kan je met behulp van dit label naar de figuur
verwijzen, met het \texttt{\bs{}ref} commando.


\begin{tcolorbox}[title=Opgave 5d]
  Voeg een caption en een label toe aan de figuur en voeg een verwijzing ernaar
  toe in de tekst, net onder de figuur.
\end{tcolorbox}

Voer \LaTeX{} \textbf{\'e\'en} maal uit en bekijk het resultaat. Klopt
de referentie naar de figuur die je toegevoegd hebt? Wellicht niet.
Als je \LaTeX{} uitvoert wordt er een lijst van labels aangemaakt en opgeslagen
in \'e\'en van de hulpbestanden.
Echter, \LaTeX{} zal die lijst pas de volgende keer als je \LaTeX{}
uitvoert gebruiken. Als je referenties gebruikt moet je dus \LaTeX{}
altijd 2 maal uitvoeren als je zeker wilt zijn dat alle referenties
correct zijn! Voer \LaTeX{} dus nogmaals uit. Het resultaat zou nu
moeten kloppen.

In de rest van de oefenzitting zullen we gebruik maken van de tool
\texttt{latexmk}. Deze tool kan automatisch detecteren hoe vaak \LaTeX{} moet
compileren en zal net zo vaak compileren met de juiste tools tot het resultaat
correct zou moeten zijn.


\begin{tcolorbox}[title=Opgave 5e]
  Voer het commando \texttt{latexmk -C} uit. Dit verwijdert alle tijdelijke
  bestanden. Dit commando kun je ook gebruiken in de map van het groepswerk en
  zal je helpen voorkomen dat je tijdelijke bestanden van \LaTeX{} gaat
  toevoegen aan de repository.
\end{tcolorbox}
\begin{tcolorbox}[title=Opgave 5f]
  Voer het commando \texttt{latexmk -pdf opgave5.tex} uit. Dit compileert je
  bestand volledig. Controleer dat ook in dit geval de referentie klopt. Dit
  commando voert op de achtergrond pdflatex uit.
\end{tcolorbox}

\subsection{Afbeeldingen}

Voor afbeeldingen worden in \LaTeX{} niet alle bestandsformaten
toegelaten. Welke formaten zijn toegelaten hangt ook af van de manier
waarop je \LaTeX{} gebruikt. Als je bijvoorbeeld \texttt{pdflatex} gebruikt
(zoals voor deze tekst de bedoeling is), dan kan je onder meer JPEG,
PNG en PDF afbeeldingen gebruiken.

Om die figuren toe te voegen moet je het \texttt{graphicx} pakket
gebruiken. Voeg dus een entry toe voor het
\texttt{graphicx} pakket; dit pakket heeft twee mogelijke opties:
\texttt{draft} indien je in je resultaat nog niet de echte figuur, maar
enkel de omringende box wilt en \texttt{final} indien de figuur mag
toegevoegd worden aan het resultaat. Met andere woorden: het pakket
wordt toegevoegd met de volgende regel (toe te voegen in de
preambule): \texttt{\bs{}usepackage[final]\{graphicx\}}.

\begin{tcolorbox}[title=Opgave 5g]
  Voeg het \texttt{graphicx} pakket toe.
\end{tcolorbox}

We willen nu de tekst in je hierboven gedefinieerde figuur vervangen
door de afbeelding \texttt{euro.png}. Nu gaan we deze afbeelding toevoegen
aan onze tekst aan de hand van het \texttt{\bs{}includegraphics}
commando. Je kan de documentatie zoeken op het internet, maar nu moet je het commando
gewoon \'e\'en argument meegeven, namelijk de naam van het
afbeeldingsbestand.
\begin{tcolorbox}[title=Opgave 5h]
  Voeg de afbeelding toe aan je bestand. Centreer de afbeelding en geef het de
  ondertekst ``Het Euro symbool''.
\end{tcolorbox}

Meestal wordt aan dit commando een breedte en hoogte meegegeven om de
ingeladen figuur te herschalen zodat ze past in je document. Je kan
deze meegeven als optionele argumenten van het \texttt{\bs{}includegraphics} commando.
Als je slechts \'e\'en van deze twee afmetingen specificeert, wordt de andere zo
berekend dat de ``aspect ratio'' (verhouding hoogte-breedte) van de afbeelding
bewaard blijft. Een voorbeeld van het voorgaande kan dus zijn:

\begin{itemize}
  \item \texttt{\bs{}includegraphics[width=280pt,height=160pt]\{mydrawing.png\}} of
  \item \texttt{\bs{}includegraphics[width=8cm]\{mydrawing.png\}}
\end{itemize}

In het eerste voorbeeld wordt het bestand \texttt{mydrawing.png} ingeladen en
wordt de scalering geforceerd op 280~punten breedte en 160~punten
hoogte. In het tweede voorbeeld wordt de figuur zo gescaleerd dat de
breedte 8~cm bedraagt en de hoogte in verhouding blijft.

Bemerk trouwens dat je de \texttt{\bs{}label} en \texttt{\bs{}ref} ook kan gebruiken bij
\texttt{chapters}, \texttt{sections} en zelfs stukken tekst.

\subsection{Inhoudstabel}

\LaTeX{} kan automatisch een inhoudstafel voor je genereren. Probeer
dit uit door het \texttt{\bs{}tableofcontents} commando toe
te voegen vlak na de aanmaak van het titelblad.  Op een
gelijkaardige manier kun je een lijst van (drijvende) figuren en een
lijst van (drijvende) tabellen toevoegen met respectievelijk het
\texttt{\bs{}listoffigures} en \texttt{\bs{}listoftables} commando.
Ook bij deze twee commando's moet je \LaTeX{} twee maal uitvoeren.
\begin{tcolorbox}[title=Opgave 5i]
  Voeg een inhoudstabel toe.
\end{tcolorbox}

\subsection{Bestanden Opsplitsen}
Wanneer je bestand groot wordt of je hieraan met meerdere mensen moet samenwerken,
gebruik je best een version control system zoals Git.
In het geval dat meerdere personen aan dezelfde tekst gewerkt hebben, zal dergelijke
software steeds zoveel mogelijk automatisch proberen te mergen. Hiervoor gaat het ervan uit
dat wanneer er slechts \'e\'en persoon wijzigingen heeft uitgevoerd rond dezelfde {\em regel}, deze zonder problemen
opgenomen kunnen worden in het samengevoegde resultaat. Wanneer echter meerdere personen
wijzigingen proberen door te voeren rond dezelfde regel, dan moet er om manuele interventie
gevraagd worden om de twee versies samen te voegen.\footnote{Bij het groepswerk ga je ongetwijfeld in een
  dergelijke situatie terecht komen. Zoek eens op wat het commando {\tt latexdiff} precies doet en test het uit.}
Het is dan ook af te raden om volledige paragrafen op eenzelfde
regel te schrijven. Beter is om deze op te delen over verschillende regels
(b.v.\ een zin per regel). Aangezien \LaTeX{} enkel een lege
regel als het einde van een alinea beschouwt, is dit geen probleem. Zo kun
je ook commentaar binnen een paragraaf schrijven.

\LaTeX{} biedt echter nog een andere manier aan om merge conflicten te vermijden.
Je kunt je \LaTeX{}-bestanden opsplitsen in meerdere delen. Zo wordt elk
afzonderlijk deeltje gemakkelijker te bewerken en kan je ook met meerdere
mensen tegelijkertijd aan eenzelfde document werken zonder continu
merge-conflicten te hebben in Git.\footnote{Dit kan zeer nuttig zijn voor jullie groepswerk!}

\LaTeX{} biedt hiervoor het \texttt{\bs{}input}-commando aan.
De werking is eenvoudig:
\begin{verbatim}
 \input{anderbestand}
\end{verbatim}
Bovenstaand commando gaat ervoor zorgen dat het bestand
\texttt{anderbestand.tex} opgenomen wordt in de verwerking van het huidige
bestand. Het is dus alsof het \texttt{\bs{}input}-commando
vervangen wordt door de volledige inhoud van \texttt{anderbestand.tex}.

\begin{tcolorbox}[title=Opgave 5j]
  Splits de twee grote secties uit de voorbeeldtekst op in twee aparte
  bestanden \texttt{sectie1.tex} en \texttt{sectie2.tex} in dezelfde directory en gebruik
  het \texttt{\bs{}input}-commando om deze in de uiteindelijke tekst te combineren.
\end{tcolorbox}

\section{\BibTeX}\label{sec:bibtex}

Bij het schrijven van artikels is het citeren van andere papers
bijzonder belangrijk. Een citaat geeft langs \'e\'en kant aan waar je
je inspiratie gehaald hebt (onder meer belangrijk om te voorkomen dat
je plagiaat pleegt!) en geeft anderzijds een startpunt aan de lezer
van je artikel indien die meer informatie wil opzoeken over het
onderwerp.

Voor het practicum van dit vak is het natuurlijk belangrijk dat in
jullie artikel de bronnen geciteerd worden die jullie gevonden en
gebruikt hebben. Daarom geven we in deze sectie een korte introductie
tot de \LaTeX{}-tool die meestal gebruikt wordt om bibliografie\"en
bij te houden: \BibTeX.

\subsection{\BibTeX-entries}

Een \BibTeX-bibliografie wordt bijgehouden in een tekstbestand met
extensie \texttt{.bib}. Dit tekstbestand bevat een lijst van
zogenaamde \BibTeX-\emph{entries}. Zulke entries refereren
typisch naar artikels, boeken of websites, maar kunnen eventueel ook
naar andere dingen verwijzen.

Elke entry heeft een aantal eigenschappen, afhankelijk van
het type. Zo heeft een verwijzing naar een boek bijvoorbeeld de
eigenschappen `titel', `ISBN-nummer', `auteur', `uitgever', etc. Een
verwijzing naar een website daarentegen heeft bijvoorbeeld geen
ISBN-nummer, maar wel een URL. Een \BibTeX-entry hoeft niet per se
alle eigenschappen die ermee geassocieerd kunnen worden op te geven
(bijvoorbeeld, bij een verwijzing naar een boek hoeft niet per se
ingevuld te worden welke editie van het boek het is), maar
natuurlijk zijn sommige eigenschappen wel belangrijk. Zo stelt een
verwijzing naar een website niet veel voor als de URL-eigenschap
niet ingevuld is.

Elke entry in een \BibTeX-bestand volgt een bepaald formaat.
Afhankelijk van het type van de entry verandert het formaat lichtjes,
maar de syntax blijft altijd dezelfde. Deze syntax ziet er uit als in
\cref{fig:bibtexstructure}.

\begin{figure}
  \begin{verbatim}
@entry_type{korte-naam,
    eigenschap-1 = {waarde},
    eigenschap-2 = {waarde-2},
    ...
    eigenschap-N = {waarde-N},
}
\end{verbatim}
  \caption{De structuur van een \BibTeX-entry}
  \label{fig:bibtexstructure}
\end{figure}

Elk type heeft een aantal eigenschappen:

\begin{description}
  \item[Vereiste eigenschappen] Deze eigenschappen zijn in principe
        vereist. Als ze niet opgegeven worden zal \BibTeX{} een waarschuwing
        genereren tijdens de compilatie van het document. Indien deze
        eigenschappen niet opgegeven worden, kan het zijn dat de
        tekst-formattering van het citaat er slecht uitziet.

  \item[Optionele eigenschappen] Deze eigenschappen mogen opgegeven
        worden, maar kunnen ook zonder probleem weggelaten worden.

  \item[Genegeerde eigenschappen] Alle eigenschapen die niet vereist
        of optioneel zijn worden genegeerd.
\end{description}

De belangrijkste types die in \BibTeX{} gebruikt kunnen worden zijn
hieronder weergegeven:\footnote{Meer informatie kan je ook vinden op \url{http://en.wikibooks.org/wiki/LaTeX/Bibliography_Management}}

\begin{description}

  \item[@article] Een artikel in een (typisch wetenschappelijk)
        magazine. Vereiste velden: author, title, journal, year. Optionele
        velden: volume, number, pages, month, note.

  \item[@book] Een boek. Vereiste velden: author
        or editor, title, publisher, year. Optionele velden: volume of
        number, series, address, edition, month, note.

  \item[@inbook] Een deel van een boek, wat een bepaald hoofdstuk kan zijn, maar ook een
        sectie of een bepaalde reeks bladzijden. Vereiste velden: author of
        editor, title, chapter en/of pages, publisher, year. Optionele
        velden: volume of number, series, type, address, edition, month,
        note.

  \item[@inproceedings] Een artikel dat gepubliceerd is op een
        wetenschappelijke conferentie. Vereiste velden: author, title,
        booktitle, year. Optionele velden: editor, volume of number, series,
        pages, address, month, organization, publisher, note.

  \item[@misc] Gebruik dit als je geen van de anderen kan gebruiken.
        Vereiste velden: geen. Optionele velden: author, title,
        howpublished, month, year, note.

  \item[@phdthesis] Een doctoraatsthesis. Vereiste velden: author, title, school,
        year. Optionele velden: type, address, month, note.

  \item[@unpublished] Een document dat niet formeel gepubliceerd is. Vereiste velden: author, title, note.
        Optionele velden: month, year.

\end{description}

\begin{figure}
  \begin{verbatim}
 @article{Dijkstra:1968:goto,
    author = {Dijkstra, Edsger W.},
    title = {Letters to the editor: go to statement considered harmful},
    journal = {Communications of the ACM},
    volume = {11},
    number = {3},
    year = {1968},
    issn = {0001-0782},
    pages = {147--148},
    doi = {http://doi.acm.org/10.1145/362929.362947},
    publisher = {ACM},
    address = {New York, NY, USA}
 }
\end{verbatim}
  \caption{Een voorbeeld \BibTeX-entry}\label{fig:bibtexexample}
\end{figure}

Een voorbeeld van een \BibTeX-entry wordt gegeven in
\cref{fig:bibtexexample}. Dit is een \BibTeX-definitie van het
bekende artikel \emph{``go to statement considered harmful''} van
Edsger Dijkstra. Als naam kiezen we `Dijkstra:1968:goto'. Deze naam
kan je zelf kiezen, maar typisch is het een combinatie van de naam
van de eerste auteur, het jaartal, en dan een sleutelwoord uit de
titel. In deze naam mogen geen komma's of spaties voorkomen.

\subsection{Gebruik vanuit \LaTeX}

In opgave 6 zie je een voorbeeld van het gebruik van \BibTeX. De tekst bestaat
uit twee bestanden:

\begin{description}
  \item[opgave6.tex] Dit is een standaard \LaTeX-bestand dat de
        tekst van het artikel bevat.

  \item[opgave6.bib] De bijhorende bibliografie. Dit bestand bevat
        drie referenties waarnaar verwezen wordt vanuit de tekst.
\end{description}

Bij het compileren van een tekst die gebruik maakt van \BibTeX{} wordt
een iteratief proces doorlopen zoals je al eerder gezien hebt bij
referenties naar figuren en tabellen. Eerst wordt de tekst van het
\LaTeX-bronbestand een eerste keer naar PDF gecompileerd. Tijdens dit
proces wordt er een \texttt{.aux}-bestand aangemaakt. In een tweede stap
wordt \BibTeX{} opgestart om op basis van de database van referenties in
je \texttt{.bib} bestand en de verwijzingen die in het document voorkomen een
geformatteerde bibliografie te genereren. Hierna wordt het document
opnieuw geformatteerd, wordt de gegenereerde bibliografie opgenomen in
het document en worden de referenties ingevuld met de juiste
verwijzingen.

Als je werkt via de console, moet je dus de volgende opdrachten uitvoeren:
\begin{center}
  \begin{minipage}{.7\linewidth}
    \begin{verbatim}
     pdflatex bestandsnaam.tex
     bibtex bestandsnaam
     pdflatex bestandsnaam.tex
     pdflatex bestandsnaam.tex
   \end{verbatim}
  \end{minipage}
\end{center}

Je kunt alternatief ook \texttt{latexmk -pdf bestandsnaam.tex} gebruiken.
\texttt{latexmk} zal herkennen dat \BibTeX{} gebruikt is en zal dit commando op
het juiste moment uitvoeren.

\begin{tcolorbox}[title=Opgave 6a]
  Compileer het bestand opgave6.tex handmatig met \texttt{pdflatex} en
  \texttt{bibtex}. Controleer na iedere stap hoe de referenties er uit zien.
\end{tcolorbox}

Een \LaTeX{}-bestand dat \BibTeX{} gebruikt moet wel op een bepaalde manier
opgeven welk \texttt{.bib} bestand het gebruikt. Daarnaast moet er ook een
bibliografiestijl opgegeven worden, die bepaalt hoe de bibliografie
geformatteerd wordt. De volgende commando's worden hiervoor gebruikt:

\begin{verbatim}
  \bibliographystyle{plain}
  \bibliography{opgave6}
\end{verbatim}

Het \texttt{\bs{}bibliographystyle}-commando gebruik je om
de stijl van de bibliografie aan te passen. Het verwacht \'e\'en
parameter die de volgende waarden kan aannemen: plain, unsrt, abbrv,
alpha. In het voorbeeld wordt het type `plain' gebruikt.
Probeer zelf de andere types en kijk wat het verschil is.
\begin{tcolorbox}[title=Opgave 6b]
  Probeer verschillende bibliografie stijlen uit.
\end{tcolorbox}

Het \texttt{\bs{}bibliography}-commando geeft aan welk
bestand je bibliografie bevat, in dit geval \texttt{opgave6.bib}. Merk op dat
de \texttt{.bib}-extensie niet opgegeven mag worden.

Eenmaal je \LaTeX-bestand gelinkt is aan een bibliografie, kan je
gaan citeren. Dit doe je door in je tekst gebruik te maken van het
\texttt{\bs{}cite}-commando. Dit commando werkt analoog aan
het \texttt{\bs{}ref}-commando, met als enig verschil dat
je nu als parameter de naam van je \BibTeX-entry ingeeft.

\subsection{Opzoeken van \BibTeX-entries}

Voor de meeste wetenschappelijke artikels kan je \BibTeX-entries
online vinden. Probeer dus zeker eerst online te zoeken naar het
\BibTeX-entry van het artikel dat je wil citeren, voordat je er zelf
\'e\'en probeert op te stellen. Zo kan je bijvoorbeeld de
\BibTeX-entries vinden van de meeste informatica-artikels op
\url{http://dblp.uni-trier.de}.

Daarnaast kan je ook gebruik maken van Google Scholar
(\url{http://scholar.google.com/}). Bij de voorkeursinstellingen kan
je aangeven dat je van de artikels ook de \BibTeX-informatie wil zien.

Kijk sowieso steeds de entries manueel na, sommige websites plaatsen
er links naar zichzelf in en soms moet je enkele dingen aanpassen om
al je referenties er consistent te laten uitzien. Tijdens een
zoektocht naar artikels rond een bepaald onderwerp is het een goed
idee om \BibTeX-entries bij te houden voor alle artikels die je
bekijkt, zodat je zeker bent dat je later alles makkelijk terugvindt.

\begin{tcolorbox}[title=Opgave 6c]
  Voeg een extra referentie toe aan de bib-file. Je zult merken dat deze nog
  niet in de bibliografie terecht komt. Citeer de referentie, en controleer
  dat de referentie nu wel in de bibliografie vermeld staat.
\end{tcolorbox}

\begin{tcolorbox}[title=Opgave 6d]
  Voor één van de referenties die je in het groepswerk gevonden hebt, zoek de
  \BibTeX-entry op en voeg deze toe aan je groepswerk repository. Zet de
  referentie in het bestand \texttt{referenties.bib}. Vergeet het bestand niet
  te adden, committen en pushen! Indien je een merge conflict krijgt, los het
  dan op zoals je vorige week geleerd hebt.
\end{tcolorbox}

Veel succes!
\flushright{}
Yolande Berbers\\
Neline van Ginkel\\
Kobe Vrancken\\
Gertjan Franken\\
Hans Winderix\\
Jan Vermaelen\\

\end{document}
