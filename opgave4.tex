\documentclass[a4paper]{article}
\usepackage[dutch]{babel}

\begin{document}
\title{Ontstaan van de Euro, criteria en deelnemers}
\author{Wouter Bouckaert}
\maketitle

\section{De realisatie van de EMU}

De weg naar de eenheidsmunt werd reeds eind jaren '80 uitgestippeld
toen de Europese Unie besloot om de economische ruimte verder te
laten groeien. Hiertoe werd in het rapport Delors in 1989 een drie
fasen plan uitgestippeld. Dit plan werd naderhand geconcretiseerd in
het Verdrag van Maastricht.

\subsection{De eerste fase (01/07/1990 - 31/12/93)}

Gedurende deze eerste fase werden de eerste, schuchtere stappen
gezet om de economische en monetaire samenwerking binnen de EU op te
drijven. Hiertoe werd begin jaren '90 het kapitaalverkeer
vrijgemaakt. Ook het vrij verkeer van goederen, diensten én personen
tussen de landen van de EU werd ingevoerd.

Het belangrijkste moment uit die periode is ongetwijfeld de
onderteking van het Verdrag van Maastricht op 7 februari 1992. Dit
ondertussen beroemde en beruchte Verdrag organiseerde de overgang
naar de EMU. Het Verdrag legde de convergentiecriteria vast. Dit
zijn de vereisten waaraan een kandidaat lidstaat moet voldoen om aan
de eenheidsmunt te mogen deelnemen. Hierop komen we straks terug.
Ook de uitertste datum (01.01.1999) voor de ingang van de derde
fase, m.a.w. de finale datum voor de invoering van de eenheidsmunt
werd op papier vastgelegd.

\subsection{De tweede fase (01/01/1994 - 31/12/1998)}

Tijdens de tweede fase hadden de diverse kandidaat lidstaten de
moelijke taak om hun economisch en vooral hun monetair beleid op
elkaar gaan afstemmen. Tevens werden de (technische) voorbereidingen
om de eenheidsmunt in te voeren uitgewerkt.

Op 1 januari 1994 werd de voorloper van de Europese Centrale Bank
(ECB), het EMI (Europees Monetair Instituut) opgericht. Aan het
hoofd stond de Belg Alexandre Lamfalussy. De opdracht van de EMI
bestond erin de nodige voorbereidingen te treffen om later de ECB in
het leven te roepen. Dit impliceerde o.a. het uitwerken van
procedures en richtlijnen om een doeltreffend monetair beleid te
voeren in de derde fase en de samenwerking tussen de centrale banken
te organiseren.

In mei 1998 werd dan de ECB opgericht. De belangrijkste doelstelling
van de ECB is het bewaren van prijsstabiliteit (lage inflatie) over
gans euroland. Aan het hoofd staat de Nederlander Wim Duisenberg.
Begin mei werden eveneens de deelnemers aan de euro geselecteerd op
basis van de economische resultaten in 1997.

In datzelfde weekend werden de onderlinge koersen van de 11
deelnemende landen vastgelegd waarin de nationale munten in de euro
zullen opgaan op 31/12/98. Besloten werd om de bilaterale
spilkoersen uit het EMS naar voor te schuiven. Er was toen evenwel
nog geen absolute zekerheid dat deze bilaterale spilkoersen de
definitieve pariteiten zouden worden. Het Verdrag van Maastricht
bepaalde namelijk dat de overgangskoersen van de nationale munten
niet mochten afwijken van de marktkoersen op de laatste
verhandelingsdag van de nationale munten. De afwijking was echter
minimaal aangezien de meeste munten van de deelnemende landen naar
die bilaterale conversiekoersen evolueerden.

\subsection{De derde fase (01/01/1999 - 31/12/2001)}

De laatste fase begon eigenlijk al op 31/12/1998 toen in de namiddag
de wisselkoersen van de lidstaten vastgelegd werden. De benaming
wisselkoersen werd sindsdien vervangen door de term
omrekeningsgetallen. Deze finalisatie van de EMU luidde de start in
van een overgangsperiode van drie jaar waar de nationale munten en
de euro naast elkaar blijven bestaan.

De kalender voor de invoering van de euro ziet er zo uit:

01.01.1999 : giraal geldverkeer in de nationale munten \'en in euro

01.01.2002 : invoering van biljetten en muntstukken, giraal geldverkeer
enkel in euro

01.07.2002 :  (ten laatste): alles in euro !

De periode tussen 1 januari 1999 en 31 december 2001 wordt de
overgangsperiode geheten. Gedurende deze periode circuleren de
nationale munten en de euro naast elkaar. Vanaf 1 januari 2002 wordt
enkel nog de euro in het giraal geldverkeer als wettig betaalmiddel
aanvaard. De tijdsspanne waarin biljetten en muntstukken in
nationale munten en in euro naast elkaar bestaan mag maximaal zes
maanden in beslag nemen. Waarschijnlijk zal deze periode in de
praktijk hoogstens enkele weken bedragen.


\subsection{De criteria van Maastricht}

Ten einde te mogen deelnemen aan de muntunie, dienden de kandidaten
hun economisch en monetair beleid op elkaar te hebben afgestemd.
Daartoe werden in het Verdrag van Maastricht een aantal vereisten
vastgesteld waaraan deelnemende landen moesten voldoen. Het zijn
deze beruchte criteria die de voorbije jaren vrijwel nooit uit het
nieuws waren. We zetten ze voor u nog eens op een rijtje:


- Prijsstabiliteit De kandidaat lidstaten moeten blijk geven van
``een hoge mate van prijsstabiliteit.'' In het Verdrag werd
gestipuleerd dat elke lidstaat het jaar vóór de evaluatie een
inflatiecijfer moet hebben dat hoogstens 1,5\% hoger ligt dan het
gemiddelde van de op dat domein best presterende (=laagste)
lidstaten. Dit criterium is imperatief.

- Wisselkoersen: Gedurende minimaal twee jaar moet het devies van de
lidstaat ``de normale fluctuatiemarges (2,25\%) van het
wisselkoersmechanisme van het Europees Monetair Stelsel (EMS I)''
zonder devaluatie t.o.v. de munt van een andere lidstaat'' naleven.
Een lidstaat waarvan de munt niet aan dit stabiliteitscriteria
beantwoord, kan per definitie niet aan de eenheidsmunt deelnemen
(cfr. Zweden).

- Rentetarieven: Tijdens het jaar dat aan de beoordeling voorafgaat
mag de nominale rentevoet voor de lange termijn, i.e. de rentevoet
van de overheidsobligaties, maximaal 2\% hoger liggen dan de
vergelijkbare rentevoet van de drie best presterende lidstaten.

- Overheidsschulden: Dit objectief is tweeledig. Enerzijds wordt een
doelstelling opgelegd voor het begrotingstekort, anderzijds wordt
ook de overheidsschuld aangepakt. We verduidelijken: De verhouding
tussen het overheidstekort en het Bruto Binnenlands Produkt (BBP)
mag hoogstens 3\% bedragen. Daarenboven mag de verhouding tussen de
overheidsschulden en het BBP niet meer bedragen dan 60\%. Deze twee
cijfers (vooral dan die 60\%) zijn echter niet imperatief, maar
``streefcijfers.''

Het Verdrag stipuleert namelijk dat ``de verhouding
begrotingstekort/BBP in aanzienlijke mate en voortdurend moet zijn
afgenomen en een niveau bereikt hebben dat de referentiewaarde
benadert.'' De overschrijding van die 3\% mag bijgevolg ``slechts
van uitzonderlijke en tijdelijke aard'' zijn en moet ``dicht bij de
referentiewaarde'' blijven.

Het objectief voor de overheidsschuld/BBP kan echter -gelukkig voor
België- soepeler worden geïnterpreteerd. De verhouding mag namelijk
de 60\% overschrijden indien ze ``in voldoende mate afneemt en de
referentiewaarde in een bevredigend tempo benaderd.''



\section{De deelnemers}

Tijdens het weekend van 2 en 3 mei 1998 werden de deelnemers aan de
eenheidsmunt geselecteerd. Het zijn volgende landen die vanaf 1
januari 1999 de euro invoeren: België, Nederland, Luxemburg,
Duitsland, Frankrijk, Italië, Spanje, Portugal, Oostenrijk, Ierland
en Finland. Dit zijn de ``ins.'' Deze landen hebben ook een
``stabiliteitspact'' afgesloten waarin ze zich engageren hun
economieën blijvend op elkaar af te stemmen.

Vier andere landen die behoren tot de EU nemen niet deel van bij de
start. Het zijn Denemarken, en het Verenigd Koninkrijk die
vrijwillig afzagen van een deelname en Zweden en Griekenland, twee
landen dat niet aan de criteria voldeden. Deze vier landen worden de
``outs.'' geheten. Deze landen kunnen eventueel in een latere fase
deelnemen. Er is hiervoor een tweejaarlijkse evaluatie voorzien,
maar vroegere toetreding is mogelijk op vraag van de kandidaat
lidstaat.

\end{document}
